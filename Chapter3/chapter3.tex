\chapter{My third chapter}

% **************************** Define Graphics Path **************************
\ifpdf
    \graphicspath{{Chapter3/Figs/}{Chapter3/Figs/PDF/}{Chapter3/Figs/}}
\else
    \graphicspath{{Chapter3/Figs/Vector/}{Chapter3/Figs/}}
\fi
You should break your thesis up into nice, bite-sized sections and subsections. \LaTeX{} automatically builds a table of Contents by looking at all the \verb|\chapter{}|, \verb|\section{}|  and \verb|\subsection{}| commands you write in the source.

The Table of Contents should only list the sections to three (3) levels. A \verb|chapter{}| is level zero (0). A \verb|\section{}| is level one (1) and so a \verb|\subsection{}| is level two (2). In your thesis it is likely that you will even use a \verb|subsubsection{}|, which is level three (3). The depth to which the Table of Contents is formatted is set within \textbf{PhDThesisPSnPDF.cls}. If you need this changed, you can do it in \textbf{Polito PhD thesis template.tex}. 

\section{First section of the third chapter}
And now I begin my third chapter here \dots


\subsection{First subsection in the first section}
\dots and some more 

\subsection{Second subsection in the first section}
\dots and some more \dots

\subsubsection{First subsub section in the second subsection}
\dots and some more in the first subsub section otherwise it all looks the same
doesn't it? well we can add some text to it \dots

\subsection{Third subsection in the first section}
\dots and some more \dots

\subsubsection{First subsub section in the third subsection}
\dots and some more in the first subsub section otherwise it all looks the same
doesn't it? well we can add some text to it and some more \dots

\subsubsection{Second subsub section in the third subsection}
\dots and some more in the second subsub section otherwise it all looks the same
doesn't it? well we can add some text to it \dots

\section{Second section of the third chapter}
and here I write more \dots

\section{In Closing}

You have reached the end of this mini-guide. You can now rename or overwrite this pdf file and begin writing the rest of your thesis. The easy work of setting up the structure and framework has been taken care of for you. It's now your job to fill it out!

Good luck and have fun!

